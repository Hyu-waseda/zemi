% \documentclass[b5paper,12pt,twocolumn]{jsreport}
\documentclass[b5paper,12pt]{jsreport}
\usepackage[top=2cm, bottom=2cm, left=2.5cm, right=2.5cm]{geometry}

\usepackage{bm}
% \usepackage[dvipdfmx]{graphicx}
\usepackage{ascmac}
% \usepackage{texcount} % 文字数カウント用 


\title{多言語自動翻訳掲示板の利活用に関する基礎的研究}
\author{早稲田大学人間科学部人間情報科学科\\
西村研究室\\
\\
学籍番号:1J20F037\\
名前:奥村飛悠}

\date{\today}

\begin{document}
\maketitle
\tableofcontents

\chapter{はじめに}


% 本研究では、異文化間の交流の増加と掲示板を通じたコミュニケーションの重要性に着目し、多言語自動翻訳掲示板の利活用に関する基礎的研究を行う。

近年、グローバル化の進展により、異なる文化や言語を持つ人々との接触がますます増えている。このような異文化間の交流は、相互理解や国際協力の促進に大きな意義を持っている。

一方で、異文化間のコミュニケーションには言語の壁が存在し、円滑なコミュニケーションが困難となることがある。また、掲示板はインターネット上での情報交換や意見共有の場として広く利用されており、異なる文化や言語を持つ人々が集まる場でもある。掲示板を通じたコミュニケーションは、自文化内のみならず異文化との交流にも役立つことが期待されている。

本研究では、多言語自動翻訳掲示板を開発し、その利活用の可能性を探求する。具体的には、掲示板をネット上に公開し、さまざまなユーザーがどのように利用するのか、異文化間のコミュニケーションがどのように行われるのかという実際のデータを収集し、分析を行う。これにより、異文化間のコミュニケーションを支援する手段として多言語自動翻訳掲示板の有用性を明らかにすることを目指す。

% 研究の目的は、異文化間の交流を促進し、相互理解を深めるために多言語自動翻訳掲示板が果たす役割を明らかにすることだ。具体的には、掲示板を利用するユーザーの属性や動機、異なる言語間のコミュニケーションの特徴、異文化間の交流がもたらす効果について、定量的および定性的なデータ分析を行う。
% 研究の目的は、掲示板を利用するユーザーの属性や動機、異なる言語間のコミュニケーションの特徴、異文化間の交流がもたらす効果について、定量的および定性的なデータ分析を行い、それらについて考察することである。
研究の目的は、異文化間のコミュニケーションを支援し、相互理解を深めるための具体的な手段を提案することです。具体的には、多言語自動翻訳掲示板の有用性や課題、ユーザーの属性や動機、異文化間の交流がもたらす効果について、定量的および定性的なデータ分析を行い、実証的な結果を導き出します。

% 研究の成果としては、多言語自動翻訳掲示板の有用性を実証することや、異文化間の交流を促進するための具体的な提案を行うことが挙げられる。これにより、異文化間のコミュニケーションの円滑化や相互理解の促進に寄与することが期待される。
\section{こっからしたテスト}

\section{現状と問題点}

最近の現状と問題点% とか。研究の成果としては、多言語自動翻訳掲示板の有用性を実証することや、異文化間の交流を促進するための具体的な提案を行うことが挙げられます。これにより、異文化間のコミュニケーションの円滑化や相互理解の促進に寄与することが期待されます。

\section{解決策の提案}

こうしたらいい,とか。

\section{数式の書き方}

アインシュタイン方程式は以下の通りである。
\begin{equation}
    R_{\mu\nu} - \frac{1}{2} g_{\mu\nu} R = 
    \frac{8\pi G}{c^2} T_{\mu\nu}
\end{equation}



\chapter{つぎに}

この辺から本番。

\section{文献の引用の仕方} 

データは参考文献\cite{rika} にあったものを使った.
この文献\cite{ten}も参考にした。

% \section{図の挿入の仕方}
% \begin{figure}[h]
%   \begin{center}
%     % \includegraphics[width=7cm]{./plot1.pdf}
%     \caption{サイン関数のグラフ}
%   \end{center}
% \end{figure}


\chapter{最後に}

結論とか,まとめとか。
最後にいうのもなんだが,ベクトルの書き方。
% \begin{itemize}
%   \item 普通の$\alpha$は\verb|\alpha|で書く。
%   \item \verb|$\vec{\alpha}$| で $\vec{\alpha}$
%   \item \verb|\usepackage{bm}| している場合は
%         \verb|$\bm{\alpha}$| で $\bm{\alpha}$
%   \item 並べると,$\alpha$, $\vec{\alpha}$, $\bm{\alpha}$
% \end{itemize}


\chapter*{謝辞}

謝辞には第何章とかの番号をつけなくてもよいので,そんなときは,
\verb|\chapter*{ }| という具合に書きます。

みなさん,ありがとう.(普通の人が見るのは,イントロと謝辞だけ... 
という説もあるから,忘れないで書く.)

\appendix
\chapter{付録があるときは}
プログラム文とかを書いてページ数を稼ぎたいときは,
以下のようにしてみます。

% \begin{verbatim}
% #include <iostream>
% using namespace std;
% int main() {  
%     for(int i = 1; i <= 5; i++) {
%         cout << "こんにちは, C++ の世界!   "  << i << endl;
%     }
%     return 0;
% }
% \end{verbatim}
% \verb|\usepackage{ascmac}|して\verb|screen| 環境を使うと,枠がつきます。
% \begin{screen}
% \begin{verbatim}
% #include <iostream>
% using namespace std;
% int main() {  
%     for(int i = 1; i <= 5; i++) {
%         cout << "こんにちは, C++ の世界!   "  << i << endl;
%     }
%     return 0;
% }
% \end{verbatim}
% \end{screen}

\begin{thebibliography}{99}
\bibitem{rika} 国立天文台編,理科年表 (丸善)
\bibitem{ten} 天文年鑑,誠文堂新光社。
\end{thebibliography}

\end{document}
